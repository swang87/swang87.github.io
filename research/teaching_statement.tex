
\documentclass[11pt]{article}

\usepackage{graphicx}
\usepackage{endfloat}
\usepackage{amssymb}
\usepackage{amsmath, mathtools, bbm, bm}
\mathtoolsset{showonlyrefs}
\usepackage{subcaption}
\usepackage{natbib}
\usepackage{hyperref}
%% THE NEXT TWO LINES INSERT THE PACKAGES FOR JASA FORMAT:
% \usepackage[default]{jasa_harvard}    %   for formatting citations in text
% \usepackage{JASA_manu}

\bibpunct{(}{)}{;}{a}{}{,}
%% CHANGING THE 'AND' IN THE HARVARD BIBLIOGRAPHY PACKAGE TO WHAT IT OUGHT TO BE
% \renewcommand{\harvardand}{and}
\topmargin=-0.5in
\textheight=9in
\textwidth=6.5in
\oddsidemargin=0in


\begin{document}


\title{Teaching Statement}
\author{Xiaofei Wang\\
Department of Statistics \\ 
Yale University, New Haven, CT 06510  \\ 
email: \texttt{xiaofei.wang@yale.edu} }

\maketitle
Teaching is one of my greatest passions. Watching a student realize his or her full potential within the course of a semester brings me a great sense of accomplishment. 
At Yale University, statistics Ph.D. candidates are offered opportunities to teach in the third and fourth years of the program. To get an opportunity to teach early on, I offered to be a teaching assistant for the only statistics summer course during both of my first two summers. I start by detailing my teaching experience, including courses taught. A number of professors served as excellent role models in these courses. Their teaching styles combined with my interactions with students and peers have greatly contributed to my teaching philosophy, which I describe afterwards. 

\section{Teaching Experience}
%
I served as a teaching fellow in the courses listed below. UG indicates an undergraduate level course, whereas G indicates a graduate level course. Course descriptions can be found at \url{http://statistics.yale.edu/courses}.
\begin{enumerate}
  \item STAT 107 - Introductory Statistics (UG, Summer 2010)
  \item STAT 107 - Introductory Statistics (UG, Summer 2011)
  \item STAT 361 - Data Analysis (G, Fall 2011)
  \item STAT 625 - Case Studies (G, Fall 2011)
  \item STAT 627 - Statistical Consulting (G, Fall 2011)
  \item STAT 365 - Data Mining and Machine Learning (G, Spring 2012)
  \item STAT 230 - Introductory Data Analysis (UG, Spring 2012)
  \item STAT 627 - Statistical Consulting (G, Spring 2012)
  \item STAT 238 - Probability and Statistics (UG, Fall 2012)
  \item STAT 230 - Introductory Data Analysis (UG, Spring 2013)
\end{enumerate}
 For the condensed summer STAT 107 course, my role primarily included grading homework assignments and hosting office hours twice a week. Because my first two summers were not as busy as the rest of the school year, I was able to carve out extra office hours for students that required additional one-on-one help. As a new teaching assistant, I found it helpful to be able to work closely with students, allowing me to pinpoint the exact concepts that gave them trouble and to understand the common areas for confusion. These experiences helped me to better anticipate the types of questions students tend to have when learning introductory statistics for the first time.

In the fall and spring semesters of the 2011 to 2012 school year, my teaching responsibilities shifted to applied statistics courses. All six courses taught in this school year were courses that required programming in \verb|R|. Aside from grading, each course required several office hour sessions each week, and occasional \verb|R| help sessions. STAT 361 taught the tools of data exploration and analysis via a series of hands-on data examples and data projects. STAT 230 was a similar, more introductory course geared towards undergraduate students. Both courses were difficult for a good number of students with no programming experience; I worked to help them learn enough \verb|R| to attain basic proficiency. More importantly, I made sure that the students did not simply treat \verb|R| as a black box of commands that, when run, does all the thinking; I made sure that the students learned to think critically about each dataset, acknowledging the pitfalls and caveats that often occur in practice when statistical methods are applied blindly. STAT 365 in the spring was taught in a similar manner -- requiring a combination of data examples to illustrate standard techniques. As this was a higher level course, there were fewer students and more students knew \verb|R|. The students were generally more independent, so only a few would attend my office hours. Therefore, instead of focusing my efforts on preparing well-organized office hour sessions, I looked for a way to reach out to a greater number of students -- when writing up homework solutions, I would give a summary of frequently-made mistakes; students found this to be helpful. 

In Spring 2013, I had the opportunity to again serve as teaching assistant for STAT 230. This time around, the course was taught in a team-based learning environment. Students were grouped into teams, working collaboratively on a different mini-project each class. Towards the end of each class, the teams presented their findings to everyone. The course also required individual end-of-semester projects. Because this was the first time a course was taught in a team-based learning environment in the statistics department, we (the instructor and the teaching assistants) faced many challenges throughout the semester. For one, we found some students relied too heavily on their teammates and were lagging behind. We adapted the course to address this problem by scheduling ten-minute one-on-one meetings with students to reinforce personal accountability. We started these meetings early on in order to make sure the students were making steady progress on their projects. 

I had an opportunity to serve as teaching assistant for a theoretical statistics course (STAT 238) in Fall 2012. The course had challenging problem sets which regularly stumped many students. During office hours, my approach was to thoroughly explain the concepts that underlay each problem. I found it helpful to illustrate the concepts from multiple perspectives and apply them to relatable examples. Students found this approach helpful. Some students took a similar approach, individually discussing their understanding of the concepts in their own words with their own examples to confirm their understanding. The open dialogue I was able to have with the students was mutually beneficial -- they improved their understanding and I learned to speak statistics multilingually, so to speak.

\section{Teaching Philosophy} % (fold)
\label{sec:teaching_philosophy}
I believe that a good instructor, most importantly, needs to know his or her audience. The design of a course should be dependent upon the class size, the class composition (undergraduate versus graduate and distribution of majors), and the students' depth of statistical knowledge. In order to ascertain this information, an incoming questionnaire or quiz would be helpful to gauge the students' statistical background as well as to get a sense of the general demographics. With these quiz results, the instructor can then tailor in-class examples to topics that are relevant and interesting to the students. The quiz will also help the instructor learn the areas of weakness that might require  extra care or attention.

Throughout the course, it is important to encourage an open channel of communication between the instructor and students. This ensures that the students remain motivated to excel and that any learning obstacle will be immediately brought to the attention of the instructor before it is too late. In my experience, students who start lagging behind may get back on track with
a little push in the right direction; students who have lagged behind for too long eventually become uninterested and stop trying. Therefore, it is important to identify students who have fallen behind as early as possible. Likewise, I believe that it is equally important for students to receive feedback from the instructor. Constructive criticism can help students do better in the future, enabling them to learn from mistakes before they become habit. Students that go above and beyond on assignments and projects also deserve extra praise in order to recognize their hard work. Students who display strong statistical abilities may benefit from extra attention to encourage them to attain their full potential.

I really enjoyed my time as a teaching assistant at Yale. In all my classes, I have received excellent feedback from students. I realize, however, that there is always room for improvement and as such, I will continue to treat each course that I teach as a learning opportunity for me.  
% section teaching_philosophy (end)

\end{document}
